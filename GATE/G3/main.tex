%iffalse
\let\negmedspace\undefined
\let\negthickspace\undefined
\documentclass[journal,12pt,onecolumn]{IEEEtran}
\usepackage{cite}
\usepackage{amsmath,amssymb,amsfonts,amsthm}
\usepackage{algorithmic}
\usepackage{multicol}
\usepackage{graphicx}
\usepackage{textcomp}
\usepackage{xcolor}
\usepackage{txfonts}
\usepackage{listings}
\usepackage{enumitem}
\usepackage{mathtools}
\usepackage{gensymb}
\usepackage{comment}
\usepackage[breaklinks=true]{hyperref}
\usepackage{tkz-euclide} 
\usepackage{listings}
\usepackage{gvv}                                        
%\def\inputGnumericTable{}                                 
\usepackage[latin1]{inputenc}                                
\usepackage{color}                                            
\usepackage{array}                                            
\usepackage{longtable}                                       
\usepackage{calc}                                             
\usepackage{multirow}                                         
\usepackage{hhline}                                           
\usepackage{ifthen}                                           
\usepackage{lscape}
\usepackage{tabularx}
\usepackage{array}
\usepackage{float}
\newtheorem{theorem}{Theorem}[section]
\newtheorem{problem}{Problem}
\newtheorem{proposition}{Proposition}[section]
\newtheorem{lemma}{Lemma}[section]
\newtheorem{corollary}[theorem]{Corollary}
\newtheorem{example}{Example}[section]
\newtheorem{definition}[problem]{Definition}
\newcommand{\BEQA}{\begin{eqnarray}}
\newcommand{\EEQA}{\end{eqnarray}}
\newcommand{\define}{\stackrel{\triangle}{=}}
\theoremstyle{remark}
\newtheorem{rem}{Remark}

% Marks the beginning of the document
\begin{document}
\bibliographystyle{IEEEtran}
\vspace{3cm}

\title{2016-CE-1-13}
\author{EE24BTECH11004 - ANKIT JAINAR}
\maketitle
\bigskip

\renewcommand{\thefigure}{\theenumi}
\renewcommand{\thetable}{\theenumi}
\setlength{\columnsep}{2.5em}
\begin{enumerate}

\item Out of the following four sentences, select the most suitable sentence with respect to grammar and usage:
\begin{enumerate}
    \item I will not leave the place until the minister does not meet me.
    \item I will not leave the place until the minister does not meet me.
    \item I will not leave the place until the minister meet me.
    \item I will not leave the place until the minister meets me.
\end{enumerate}

\item A rewording of something written or spoken is a 
\begin{enumerate}
    \item paraphrase
    \item paradox
    \item paradigm
    \item paraffin
\end{enumerate}

\item Archimedes said, ``Give me a lever long enough and a fulcrum on which to place it, and I will move the world.''

The sentence above is an example of a statement.
\begin{enumerate}
    \item figurative
    \item collateral
    \item literal
    \item figurine
\end{enumerate}

\item If `reltfaga' means carefree, `otaga' means careful and `fertaga' means careless, which of the following could mean `aftercare'?

\begin{enumerate}
    \item zentaga
    \item tagaf
    \item tagazen
    \item relffer
\end{enumerate}

\item A cube is built using $64$ cubic blocks of side one unit. After it is built, one cubic block is removed from every corner of the cube. The resulting surface area of the body (in square units) after the removal is 
\begin{enumerate}
    \item $56$
    \item $64$
    \item $72$
    \item $96$
\end{enumerate}

\item A shaving set company sells 4 different types of razors: Elegance, Smooth, Soft, and Executive. Sales in each quarter for each type of razor are shown in the table. The question is to find which product contributes the greatest fraction to the revenue of the company in that year.
\[
\begin{array}{|c|c|c|c|c|}
\hline
\text{Quarter} & \text{Elegance} & \text{Smooth} & \text{Soft} & \text{Executive} \\
\hline
Q1 & 21334 & 20121 & 19544 & 10342 \\
Q2 & 19075 & 19342 & 18555 & 10543 \\
Q3 & 22475 & 22234 & 20432 & 13434 \\
Q4 & 20563 & 21013 & 20732 & 12534 \\
\hline
\end{array}
\]
\begin{enumerate}
    \item[(A)] Elegance
    \item[(B)] Executive
    \item[(C)] Smooth
    \item[(D)] Soft
\end{enumerate}

\item Indian currency notes show the denomination indicated in at least seventeen languages. If this is not an indication of the nation`s diversity, nothing else could be. Which of the following can be logically inferred from the above sentences?
\begin{enumerate}
    \item[(A)] Linguistic pluralism is the only indicator of a nation`s diversity.
    \item[(B)] India has a very high linguistic diversity.
    \item[(C)] Indian currency notes have sufficient space for all the Indian languages.
    \item[(D)] Linguistic pluralism is strong evidence of India`s diversity.
\end{enumerate}
\item Consider the following statements relating to the level of play of four players $P$, $Q$, $R$, and $S$.
\begin{enumerate}
    \item $P$ always beats $Q$
    \item $R$ always beats $S$
    \item $S$ always loses to $P$
    \item $R$ always loses to $Q$
\end{enumerate}
Which of the following can be logically inferred from the above statements?
\begin{enumerate}
    \item[(A)] $P$ is likely to beat all the other three players.
    \item[(B)] $S$ is the absolute worst player in the set.
    \item[(C)] $P$ is likely to beat $R$.
    \item[(D)] Both (A) and (B).
\end{enumerate}
\item If $f(x) = 2x^3 + 3x^2 + x + 5$, which of the following is a factor of $f(x)$?
\begin{enumerate}
    \item[(A)] $(2x - 5)$
    \item[(B)] $(x - 1)$
    \item[(C)] $(2x + 3)$
    \item[(D)] $(x + 1)$
\end{enumerate}
\item In a process, the number of cycles to failure decreases exponentially with increased stress. A component fails at 10,000 cycles for a load of 70\% of its capacity. When the load is increased to 90\% of the capacity, the number of cycles to failure will happen in 5000 cycles. The load for which this will happen in 9000 cycles is .
\begin{enumerate}
    \item[(A)] 40.0\%
    \item[(B)] 46.2\%
    \item[(C)] 60.1\%
    \item[(D)] 92.0\%
\end{enumerate}

    \item Newton Raphson method is to be used to find the root of the equation $3x - e^x + \sin x = 0$. If the initial trial value for the root is taken as $0.333$, the next approximation for the root would be (note: answer up to three decimal places).
    
\item The type of partial differential equation 
    $ \frac{\partial^2 P}{\partial x^2} + \frac{\partial^2 P}{\partial y^2} + 3 \frac{\partial^2 P}{\partial x \partial y} + 2 \frac{\partial P}{\partial x} - \frac{\partial P}{\partial y} = 0 $
    is
    \begin{enumerate}
        \item[(A)] elliptic
        \item[(B)] parabolic
        \item[(C)] hyperbolic
        \item[(D)] none of these
    \end{enumerate}
    
    \item If the entries in each column of a square matrix $M$ add up to $1$, then an eigenvalue of $M$ is
    \begin{enumerate}
        \item[(A)] $4$
        \item[(B)] $3$
        \item[(C)] $2$
        \item[(D)] $1$
    \end{enumerate}
    
\end{enumerate}










\end{document}e

%iffalse
\let\negmedspace\undefined
\let\negthickspace\undefined
\documentclass[journal,12pt,onecolumn]{IEEEtran}
\usepackage{cite}
\usepackage{amsmath,amssymb,amsfonts,amsthm}
\usepackage{algorithmic}
\usepackage{multicol}
\usepackage{graphicx}
\usepackage{textcomp}
\usepackage{xcolor}
\usepackage{txfonts}
\usepackage{listings}
\usepackage{enumitem}
\usepackage{mathtools}
\usepackage{gensymb}
\usepackage{comment}
\usepackage[breaklinks=true]{hyperref}
\usepackage{tkz-euclide} 
\usepackage{listings}
\usepackage{gvv}                                        
%\def\inputGnumericTable{}                                 
\usepackage[latin1]{inputenc}                                
\usepackage{color}                                            
\usepackage{array}                                            
\usepackage{longtable}                                       
\usepackage{calc}                                             
\usepackage{multirow}                                         
\usepackage{hhline}                                           
\usepackage{ifthen}                                           
\usepackage{lscape}
\usepackage{tabularx}
\usepackage{array}
\usepackage{float}
\newtheorem{theorem}{Theorem}[section]
\newtheorem{problem}{Problem}
\newtheorem{proposition}{Proposition}[section]
\newtheorem{lemma}{Lemma}[section]
\newtheorem{corollary}[theorem]{Corollary}
\newtheorem{example}{Example}[section]
\newtheorem{definition}[problem]{Definition}
\newcommand{\BEQA}{\begin{eqnarray}}
\newcommand{\EEQA}{\end{eqnarray}}
\newcommand{\define}{\stackrel{\triangle}{=}}
\theoremstyle{remark}
\newtheorem{rem}{Remark}

% Marks the beginning of the document
\begin{document}
\bibliographystyle{IEEEtran}
\vspace{3cm}

\title{bf{2010-XE-1-13}}
\author{EE24BTECH11004 - ANKIT JAINAR}
\maketitle
\bigskip

\renewcommand{\thefigure}{\theenumi}
\renewcommand{\thetable}{\theenumi}
\setlength{\columnsep}{2.5em}


\section*{Questions}

\begin{enumerate}
\item The question below consists of a pair of related words followed by four pairs of words. Select the pair that best expresses the relation in the original pair. \\
    {Unemployed : Worker}
    \begin{enumerate}
        \item $  fallow : land $
        \item $  unwary : sleeper $
        \item $  wit : jester $
        \item $  renovated : house $
    \end{enumerate}

    \item Choose the most appropriate word from the options given below to complete the following sentence: \\
    {His rather casual remarks on politics \_\_\_\_ his lack of seriousness about the subject.}
    \begin{enumerate}
        \item $  {masked} $
        \item $  {belied} $
        \item $  {betrayed} $
        \item $  {suppressed} $
    \end{enumerate}

\item Which of the following options is the closest in meaning to the word below: \\
    {Circuitous}
    \begin{enumerate}
        \item $  {cyclic} $
        \item $  {indirect} $
        \item $  {confusing} $
        \item $  {crooked} $
    \end{enumerate}

\item 25 persons are in a room. 15 of them play hockey, 17 of them play football and 10 of them play both hockey and football. Then the number of persons playing neither hockey nor football is:
    \begin{enumerate}
        \item $  2 $
        \item $  17 $
        \item $  13 $
        \item $  3 $
    \end{enumerate}

    \item Choose the most appropriate word from the options given below to complete the following sentence: \\
    {If we manage to \_\_\_\_ our natural resources, we would leave a better planet for our children.}
    \begin{enumerate}
        \item $  {uphold} $
        \item $  {restrain} $
        \item $  {cherish} $
        \item $  {conserve} $
    \end{enumerate}

    \item 5 skilled workers can build a wall in 20 days; 8 semi-skilled workers can build a wall in 25 days; 10 unskilled workers can build a wall in 30 days. If 14 men (2 skilled, 6 semi-skilled, and 5 unskilled workers) now work, how long will it take to build the wall?
    \begin{enumerate}
        \item $  20 $ days 
        \item $  18 $ days 
        \item $  16$  days 
        \item $  15$days 
    \end{enumerate}

    \item Given digits 2, 2, 3, 3, 3, 4, 4, 4, 4, how many distinct 4-digit numbers greater than 3000 can be formed?
    \begin{enumerate}
        \item $  50 $
        \item $  51 $
        \item $ 52 $
        \item $ 54 $
    \end{enumerate}
\item If $137 + 276 = 435$ how much is $731 + 672$?
\begin{enumerate}
    \item $534$
    \item $1403$
    \item $1623$
    \item $1513$
\end{enumerate}

\item Hari (H), Gita (G), Irfan (I) and Saira (S) are siblings (i.e., brothers and sisters). All were born on 1st January. The age difference between any two successive siblings (that is born one after another) is less than 3 years. Given the following facts:
\begin{enumerate}
    \item[i.] Hari age + Gita age $> $ Irfan  age + Saira age.
    \item[ii.] The age difference between Gita and Saira is 1 year. However, Gita is not the oldest and Saira is not the youngest.
    \item[iii.] There are no twins.
\end{enumerate}
In what order were they born (oldest first)?
\begin{enumerate}
    \item HSIG
    \item SGHI
    \item IGSH
    \item IHSG
\end{enumerate}

\item Modern warfare has changed from large scale clashes of armies to suppression of civilian populations. Chemical agents that do their work silently appear to be suited to such warfare; and regretfully, there exist people in military establishments who think that chemical agents are useful tools for their cause.

Which of the following statements best sums up the meaning of the above passage:
\begin{enumerate}
    \item Modern warfare has resulted in civil strife.
    \item Chemical agents are useful in modern warfare.
    \item Use of chemical agents in warfare would be undesirable.
    \item People in military establishments like to use chemical agents in war.
\end{enumerate}

\item Given digits $2, 2, 3, 3, 3, 4, 4, 4, 4,$ how many distinct $4$-digit numbers greater than$ 3000 $can be formed?
\begin{enumerate}
    \item $50$
    \item $51$
    \item $52$
    \item $54$
\end{enumerate}
\item If $P = \begin{pmatrix} 1 & 1 \\ 1 & 1 \end{pmatrix}$, then $P^8 - 2P^7 + 2P^6 - 4P^5 + 3P^4 - 6P^3 + 2P^2$ equals
\begin{enumerate}
    \item $P$
    \item $2P$
    \item $3P$
    \item $4P$
\end{enumerate}

\item Which one of the following matrices has the same eigenvalues as that of $\begin{pmatrix} 1 & 2 \\ 4 & 3 \end{pmatrix}$?
\begin{enumerate}
    \item $\begin{pmatrix} 3 & 4 \\ 1 & 2 \end{pmatrix}$
    \item $\begin{pmatrix} 1 & 4 \\ 2 & 3 \end{pmatrix}$
    \item $\begin{pmatrix} 4 & 2 \\ 1 & 3 \end{pmatrix}$
    \item $\begin{pmatrix} 2 & 4 \\ 1 & 3 \end{pmatrix}$
\end{enumerate}

\end{enumerate}
\end{document}

%iffalse
\let\negmedspace\undefined
\let\negthickspace\undefined
\documentclass[journal,12pt,twocolumn]{IEEEtran}
\usepackage{cite}
\usepackage{amsmath,amssymb,amsfonts,amsthm}
\usepackage{algorithmic}
\usepackage{graphicx}
\usepackage{textcomp}
\usepackage{xcolor}
\usepackage{txfonts}
\usepackage{listings}
\usepackage{enumitem}
\usepackage{mathtools}
\usepackage{gensymb}
\usepackage{comment}
\usepackage[breaklinks=true]{hyperref}
\usepackage{tkz-euclide}
\usepackage{listings}
\usepackage{gvv}                                        
%\def\inputGnumericTable{}                                
\usepackage[latin1]{inputenc}                                
\usepackage{color}                                            
\usepackage{array}                                            
\usepackage{longtable}                                      
\usepackage{calc}                                            
\usepackage{multirow}                                        
\usepackage{hhline}                                          
\usepackage{ifthen}                                          
\usepackage{lscape}
\usepackage{tabularx}
\usepackage{array}
\usepackage{float}


\newtheorem{theorem}{Theorem}[section]
\newtheorem{problem}{Problem}
\newtheorem{proposition}{Proposition}[section]
\newtheorem{lemma}{Lemma}[section]
\newtheorem{corollary}[theorem]{Corollary}
\newtheorem{example}{Example}[section]
\newtheorem{definition}[problem]{Definition}
\newcommand{\BEQA}{\begin{eqnarray}}
\newcommand{\EEQA}{\end{eqnarray}}
\newcommand{\define}{\stackrel{\triangle}{=}}
\theoremstyle{remark}
\newtheorem{rem}{Remark}

% Marks the beginning of the document
\begin{document}
\bibliographystyle{IEEEtran}
\vspace{3cm}

\title{JEE ADVANCED}
\author{ee24btech11004 - ANKIT JAINAR}
\maketitle
\newpage
\bigskip

\renewcommand{\thefigure}{\theenumi}
\renewcommand{\thetable}{\theenumi}
\section{MCQs with One or More than One Correct}
\begin{enumerate}
\item The minimum value of expression sin($\alpha$) + sin($\beta$) + sin($\gamma$), where ($\alpha,\beta,\gamma$) are real numbers satisfying ($\alpha+\beta+\gamma$)=$\pi$  is \hfill(1995)\
\begin{enumerate}
    \item positive
    \item 0
    \item negative 
    \item -3
\end{enumerate}
\item The number of values of x in the interval [0,5$\pi$] satisfying equation 3 sin($^2x$)-7$sinx$+2=0 (1998-2 Marks) \
\begin{enumerate}
    \item 0
    \item 5
    \item 6
    \item 10
\end{enumerate}

\item Which of the following number(s) is/are rational? \hfill(1998-2 Marks) \
\begin{enumerate}
    \item sin15\degree
    \item cos15\degree
    \item sin15\degree cos15\degree
    \item sin15\degree cos75\degree
\end{enumerate}
\item For a positive integer n, let \ 
$f_(n)(\theta) = (\tan\frac{\theta}{2})(1+\sec\theta)(1+\sec2\Theta)(1+\sec4\theta)\dots(1+\sec2^n\theta).$ Then (1999 - 3Marks)
\begin{enumerate}
    \item $f_2(\frac{\pi}{16}) = 1$
    \item $f_3(\frac{\pi}{32}) = 1$
    \item $f_4(\frac{\pi}{64}) = 1$
    \item $f_5(\frac{\pi}{128}) = 1$
\end{enumerate}
\item If $\frac{sin^4x}{2}+\frac{cos^4x}{3}=\frac{1}{5}$ , Then \hfill(2009) \
\begin{enumerate}
    \item $tan^2x=\frac{2}{3}$
    \item $\frac{sin^8x}{8}+\frac{cos^8x}{27}=\frac{1}{125}$
    \item $tan^2x=\frac{1}{3}$
    \item $\frac{sin^8x}{8}+\frac{cos^8x}{27}=\frac{2}{125}$
\end{enumerate}
\item For $ 0< \theta <\frac{\pi}{2}$, the solution(s)of $\sum_{m=1}^{6}\cosec(\theta+\frac{(m-1)\pi}{4})\cosec(\theta\frac{m\pi}{4}) = 4\sqrt{2}$ is(are) \hfill(2009) \
\begin{enumerate}
    \item $\frac{\pi}{4}$
    \item $\frac{\pi}{6}$
    \item $\frac{\pi}{12}$
    \item $\frac{5\pi}{12}$
\end{enumerate}

\item Let $\theta$,$\phi \in[0,2\pi]$ be such that$2 cos\theta(1-\sin\phi)= 
\sin^2\theta(\tan\frac{\theta}{2}+\cot\frac{\theta}{2})\cos\phi-1$,$\tan(2\pi-\theta)>0 and -1<\sin\theta<-\frac{\sqrt{3}}{2}$,$ then \phi cannot satisfy$ \hfill(2012)
\begin{enumerate}
    \item $0<\phi<\frac{\pi}{2}$
    \item $\frac{\pi}{2}<\phi<\frac{4\pi}{3}$
    \item $\frac{4\pi}{3}<\phi<\frac{3\pi}{2}$
    \item $\frac{3\pi}{2}<\phi<2\pi$
\end{enumerate}
\item The number of points in $(-\infty, \infty)$, for which $x - x \sin x - \cos x = 0$, is 
\begin{enumerate}
    \item $6$
    \item $4$
    \item $2$
    \item $0$
\end{enumerate}
\item Let $f(x)=x\sin\pi x $,$ x>o $.Then for all  natural numbers n,f'(x)vanishes at
\hfill(JEE Adv. 2013)
\begin{enumerate}
    \item A unique point in the interval$(n,n+\frac{1}{2})$
    \item A unique point in the interval$(n+\frac{1}{2},n+1)$
    \item A unique point in the interval$(n,n+1)$
    \item Two points in the interval$(n,n+1)$
\end{enumerate}
\item Let $\alpha$ and $\beta$ be non-zero real numbers such that 2$(\cos \beta - \cos \alpha)$+$\cos \alpha \cos \beta=1$.Then which of the following is/are true? \hfill(JEE Adv.2017)
\begin{enumerate}
    \item $\tan(\frac{\alpha}{2})+\sqrt{3}\tan(\frac{\beta}{2})=0$
    \item $\sqrt{3}(\tan\frac{\alpha}{2})+\tan(\frac{\beta}{2})=0$
    \item $\tan(\frac{\alpha}{2})-\tan(\frac{\beta}{2})=0$
    \item $\sqrt{3}\tan(\frac{\alpha}{2})-\tan(\frac{\beta}{2})=0$
\end{enumerate}

\section{Subjective Problems}
\item If $\tan\alpha=\frac{m}{m+1}$ and $\tan\beta=\frac{1}{2m+1}$, find the possible values  of $(\alpha+\beta)$ \hfill(1978)
\item 
\begin{enumerate}
    \item Draw the graph of $y=\frac{1}{\sqrt{2}}$ $(\sin x+\cos x)$ from $x=-\frac{\pi}{2}$ to $x=\frac{\pi}{2}$
    \item If $\cos(\alpha+\beta)=\frac{4}{5}$,$\sin(\alpha-\beta)=\frac{5}{13}$, and $\alpha,\beta$ lies between 0 and $\frac{\pi}{4}$, find $\tan2\alpha$ \hfill(1979)
\end{enumerate}
\item Given $\alpha+\beta-\gamma=\pi$, prove that $\sin^2\alpha+\sin^2\beta-\sin^2\gamma=2\sin\alpha\sin\beta\cos\gamma$ \hfill(1980)
\item Given A=\{x:$\frac{\pi}{6} \le x \le\frac{\pi}{3}$\} and f(x)=$\cos x-x(1+x)$; find f(A) \hfill(1980)

\item For all $\theta$ in $\left[0, \frac{\pi}{2}\right]$ show that, $\cos(\sin\theta)\geq
\sin(\cos\theta).$ \hfill(1981-4 Marks)

\end{enumerate}

\end{document}

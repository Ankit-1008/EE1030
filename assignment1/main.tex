%iffalse
\let\negmedspace\undefined
\let\negthickspace\undefined
\documentclass[journal,12pt,onecolumn]{IEEEtran}
\usepackage{cite}
\usepackage{amsmath,amssymb,amsfonts,amsthm}
\usepackage{algorithmic}
\usepackage{graphicx}
\usepackage{textcomp}
\usepackage{xcolor}
\usepackage{txfonts}
\usepackage{listings}
\usepackage{enumitem}
\usepackage{mathtools}
\usepackage{gensymb}
\usepackage{comment}
\usepackage[breaklinks=true]{hyperref}
\usepackage{tkz-euclide}
\usepackage{listings}
\usepackage{gvv}                                        
%\def\inputGnumericTable{}                                
\usepackage[latin1]{inputenc}                                
\usepackage{color}                                            
\usepackage{array}                                            
\usepackage{longtable}                                      
\usepackage{calc}                                            
\usepackage{multirow}                                        
\usepackage{hhline}                                          
\usepackage{ifthen}                                          
\usepackage{lscape}
\usepackage{tabularx}
\usepackage{array}
\usepackage{float}


\newtheorem{theorem}{Theorem}[section]
\newtheorem{problem}{Problem}
\newtheorem{proposition}{Proposition}[section]
\newtheorem{lemma}{Lemma}[section]
\newtheorem{corollary}[theorem]{Corollary}
\newtheorem{example}{Example}[section]
\newtheorem{definition}[problem]{Definition}
\newcommand{\BEQA}{\begin{eqnarray}}
\newcommand{\EEQA}{\end{eqnarray}}
\newcommand{\define}{\stackrel{\triangle}{=}}
\theoremstyle{remark}
\newtheorem{rem}{Remark}

% Marks the beginning of the document
\begin{document}
\bibliographystyle{IEEEtran}
\vspace{3cm}

\title{JEE ADVANCED}
\author{ee24btech11004 - ANKIT JAINAR}
\maketitle

\bigskip

\renewcommand{\thefigure}{\theenumi}
\renewcommand{\thetable}{\theenumi}
\section{MCQs with One or More than One Correct}
\begin{enumerate}
\item The minimum value of expression $\sin{\alpha} + \sin{\beta} + \sin{\gamma}$, where $\brak{\alpha,\beta,\gamma}$ are real numbers satisfying $\brak{\alpha+\beta+\gamma}=\pi$  is \hfill{(1995)}
\begin{enumerate}
    \item positive
    \item $0$
    \item negative 
    \item $-3$
\end{enumerate}
\item The number of values of $x$ in the interval $\sbrak{0,5\pi}$ satisfying equation \\$3 \sin{\brak{x^2}}-7 \sin{x+2}=0$  \hfill{(1998-2 Marks)} 
\begin{enumerate}
    \item $0$
    \item $5$
    \item $6$
    \item $10$
\end{enumerate}

\item Which of the following number(s) is/are rational? \hfill{(1998-2 Marks)} 
\begin{enumerate}
    \item $\sin{15}\degree$
    \item $\cos{15}\degree$
    \item $\sin{15}\degree \cos{15}\degree$
    \item $\sin{15}\degree \cos{75}\degree$
\end{enumerate}
\item For a positive integer $n$, let \ 
$f_n\brak{\theta} = \brak{\tan\frac{\theta}{2}}\brak{1+\sec{\theta}}\brak{1+\sec{2\theta}}\brak{1+\sec4\theta}\dots\brak{1+\sec2^n{\theta}}.$ \\Then  \hfill{(1999 - 3Marks)}
\begin{enumerate}
    \item $f_2\brak{\frac{\pi}{16}} = 1$
    \item $f_3\brak{\frac{\pi}{32}} = 1$
    \item $f_4\brak{\frac{\pi}{64}} = 1$
    \item $f_5\brak{\frac{\pi}{128}} = 1$
\end{enumerate}
\item If $\frac{\sin^4{x}}{2}+\frac{\cos^4{x}}{3}=\frac{1}{5}$ , Then \hfill{(2009)} 
\begin{enumerate}
    \item $\tan^2{x}=\frac{2}{3}$
    \item $\frac{\sin^8{x}}{8}+\frac{\cos^8{x}}{27}=\frac{1}{125}$
    \item $\tan^2{x}=\frac{1}{3}$
    \item $\frac{\sin^8{x}}{8}+\frac{\cos^8{x}}{27}=\frac{2}{125}$
\end{enumerate}
\item For $ 0<\theta <\frac{\pi}{2}$, the solution(s) of $\sum_{m=1}^{6}\cosec{\brak{\theta+\frac{(m-1)\pi}{4}}}\cosec{{\brak{\theta}+\frac{m\pi}{4}} }= 4\sqrt{2}$ is(are) \hfill{(2009)}
\begin{enumerate}
    \item $\frac{\pi}{4}$
    \item $\frac{\pi}{6}$
    \item $\frac{\pi}{12}$
    \item $\frac{5\pi}{12}$
\end{enumerate}

\item Let $\theta, \varphi \in [0,2\pi]$ be such that $2 \cos\brak{\theta\brak{1-\sin \varphi}}= 
\sin^2\brak{\theta\brak{\tan\frac{\theta}{2}}+\cot\frac{\theta}{2}}\cos \varphi-1$,$\tan\brak{2\pi-\theta}>0$ and $-1<\sin{\theta}<-\frac{\sqrt{3}}{2}$, then $\varphi$ cannot satisfy \hill{(2012)}
\begin{enumerate}
    \item $0<\varphi<\frac{\pi}{2}$
    \item $\frac{\pi}{2}<\varphi<\frac{4\pi}{3}$
    \item $\frac{4\pi}{3}<\varphi<\frac{3\pi}{2}$
    \item $\frac{3\pi}{2}<\varphi<2\pi$
\end{enumerate}
\item The number of points in $\brak{-\infty, \infty}$, for which $x - x \sin x - \cos x = 0$, is \hfill{(JEE Adv.2013)}
\begin{enumerate}
    \item $6$
    \item $4$
    \item $2$
    \item $0$
\end{enumerate}
\item Let $f\brak{x}=x\sin\pi x $, $ x>0 $. Then for all  natural numbers n, \( f^\prime(x) \) vanishes at 
\hfill{(JEE Adv. 2013)}
\begin{enumerate}
    \item A unique point in the interval $\brak{n,n+\frac{1}{2}}$
    \item A unique point in the interval $\brak{n+\frac{1}{2},n+1}$
    \item A unique point in the interval $\brak{n,n+1}$
    \item Two points in the interval $\brak{n,n+1}$
\end{enumerate}
\item Let $\alpha$ and $\beta$ be non-zero real numbers such that 2$\brak{\cos \beta - \cos \alpha}$+$\cos \alpha \cos \beta=1$.Then which of the following is/are true? \hfill{(JEE Adv.2017)}
\begin{enumerate}
    \item $\tan{\brak{\frac{\alpha}{2}}+\sqrt{3}\tan\brak{\frac{\beta}{2}}}=0$
    \item $\sqrt{3}\brak{\tan{\frac{\alpha}{2}}}+\tan\brak{{\frac{\beta}{2}}}=0$
    \item $\tan{\brak{\frac{\alpha}{2}}}-\tan{\brak{\frac{\beta}{2}}}=0$
    \item $\sqrt{3}\tan{\brak{\frac{\alpha}{2}}}-\tan{\brak{\frac{\beta}{2}}}=0$
\end{enumerate}
\end{enumerate}
\section{Subjective Problems}
\begin{enumerate}
\item If $\tan{\alpha}=\frac{m}{m+1}$ and $\tan{\beta}=\frac{1}{2m+1}$, find the possible values of $\brak{\alpha+\beta}$ \hfill{(1978)}
    \item Draw the graph of $y=\frac{1}{\sqrt{2}}$ $\brak{\sin {x}+\cos {x}}$ from $x=-\frac{\pi}{2}$ to $x=\frac{\pi}{2}$
    \item If $\cos{\brak{\alpha+\beta}}=\frac{4}{5}$,$\sin{\brak{\alpha-\beta}}=\frac{5}{13}$, and $\alpha,\beta$ lies between $0$ and $\frac{\pi}{4}$, find $\tan{2\alpha}$ \hfill{(1979)}
\item Given $\alpha+\beta-\gamma=\pi$, prove that $\sin^2{\alpha}+\sin^2{\beta}-\sin^2{\gamma}=2\sin{\alpha}\sin{\beta}\cos{\gamma}$ \hfill{(1980)}
\item Given $A=\cbrak{x:\frac{\pi}{6}\le x\le\frac{\pi}{3}}$ and f(x)=$\cos{x-x\brak{1+x}}$; find $f(A)$ \hfill{(1980)}

\item For all $\theta$ in $\brak{0, \frac{\pi}{2}}$ show that, $\cos\brak{\sin\theta}\geq
\sin{\brak{\cos{\theta}}}.$ \hfill{(1981-4 Marks)}
\end{enumerate}


\end{document}




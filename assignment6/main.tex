%iffalse
\let\negmedspace\undefined
\let\negthickspace\undefined
\documentclass[journal,12pt,onecolumn]{IEEEtran}
\usepackage{cite}
\usepackage{amsmath,amssymb,amsfonts,amsthm}
\usepackage{algorithmic}
\usepackage{multicol}
\usepackage{graphicx}
\usepackage{textcomp}
\usepackage{xcolor}
\usepackage{txfonts}
\usepackage{listings}
\usepackage{enumitem}
\usepackage{mathtools}
\usepackage{gensymb}
\usepackage{comment}
\usepackage[breaklinks=true]{hyperref}
\usepackage{tkz-euclide} 
\usepackage{listings}
\usepackage{gvv}                                        
%\def\inputGnumericTable{}                                 
\usepackage[latin1]{inputenc}                                
\usepackage{color}                                            
\usepackage{array}                                            
\usepackage{longtable}                                       
\usepackage{calc}                                             
\usepackage{multirow}                                         
\usepackage{hhline}                                           
\usepackage{ifthen}                                           
\usepackage{lscape}
\usepackage{tabularx}
\usepackage{array}
\usepackage{float}
\newtheorem{theorem}{Theorem}[section]
\newtheorem{problem}{Problem}
\newtheorem{proposition}{Proposition}[section]
\newtheorem{lemma}{Lemma}[section]
\newtheorem{corollary}[theorem]{Corollary}
\newtheorem{example}{Example}[section]
\newtheorem{definition}[problem]{Definition}
\newcommand{\BEQA}{\begin{eqnarray}}
\newcommand{\EEQA}{\end{eqnarray}}
\newcommand{\define}{\stackrel{\triangle}{=}}
\theoremstyle{remark}
\newtheorem{rem}{Remark}

% Marks the beginning of the document
\begin{document}
\bibliographystyle{IEEEtran}
\vspace{3cm}

\title{\textbf{6-04-2024 Shift-1}}
\author{EE24BTECH11004 - ANKIT JAINAR}
\maketitle
\bigskip

\renewcommand{\thefigure}{\theenumi}
\renewcommand{\thetable}{\theenumi}
\setlength{\columnsep}{2.5em}
\begin{enumerate}
\item $I = \int_{0}^{\frac{\pi}{4}} \frac{\cos^2 x \sin^2 x}{\left( \cos^3 x + \sin^3 x \right)^2} \, dx$
\item An equilateral triangle of side $12$. A circle is embedded inside the triangle, and a square is embedded inside the circle. If the area and perimeter of the square are $m$ and $n$, respectively, then find $m + n^2$.
\item Solve: $\left(1 + x^2 \right) \frac{dy}{dx} + y = e^{\tan^{-1}x}, \, y(1) = 0$. Then $y(0) =$
\item Find the range of $x$ for which $f(x) = x^x \, (x > 0)$ is strictly increasing.
\item Let $A = \{100, 101, 102, \dots, 700\}$. Find the number of numbers in set $A$ which are neither divisible by 3 nor by 4.
\item Given that $\frac{dy}{dx} + 2x \ln{x} \cdot y = 3 \ln{x}$, and $y(1) = 0$, find $y$.
\item Let 
$A_r = \begin{vmatrix}
r & 1 & \frac{n^2}{2} + \alpha \\
2r & 2 & n^2 - \beta \\
3r - 2 & 3 & n(n-1)
\end{vmatrix}$
Find $2A_{10} - A_8$.
\item In an octagon how many triangles are possible so that no side of triangle is side of octagon?
\item A variable line is passing through \brak{4, -9}, slope of line is positive and it make intercepts on
x and y-axis on point A and B. Find the minimum area of triangle OAB.
\item If mean of $20$ observation is$ 10$, $SD = 2$. One of the observation which is $12$ is replaced by$ 8$.
Find the value of new $SD$?
\item Let $f : \mathbb{R} \to \mathbb{R}$ be defined by $f(x) = \frac{x^2 - 2x - 15}{x^2 - 4x + 9}$, then $f$ is:
    \begin{enumerate}
        \item one-one onto
        \item many-one onto
        \item many-one into
        \item one-one into
    \end{enumerate}
\item A company has two branches $A$ and $B$. Branch $A$ produces $60\%$ of the total production and the remaining by branch $B$. Branch $A$ produces $80\%$ good quality products, and branch $B$ produces $90\%$ good quality products. A product is randomly selected, and it is found to be of good quality. Let $P$ be the probability that the selected product is from branch $B$. Find the value of $126P$.
\item Find the shortest distance between two lines:
$\frac{x - 3}{2} = \frac{y + 15}{-7} = \frac{z - 9}{5} \quad and \quad \frac{x + 1}{2} = \frac{y - 1}{1} = frac{z-9}{-3}.$
\item If in the expansion of  $(x + y)^n$, the terms are:
$T_2 = 15, \quad T_3 = 10, \quad T_4 = \frac{10}{3}$ For $n = 5$,  find the value of  $n^3+x^5+243y^5$.
\item Let $S = \{1, 2, 3, \dots, 20\}$ be a given set. Relation $R_1$ is defined as $R_1 = \{(x, y) : 2x - 3y = 2\}$ and $R_2$ as $R_2 = \{(x, y) : 4x = 5y\}$, where $x, y \in S$. If $m$ denotes the number of elements required to make $R_1$ symmetric, and $n$ denotes the number of elements to make $R_2$ symmetric,find $m+n$.
\end{enumerate}
\end{document}
